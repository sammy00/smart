%!TEX program=xelatex
\documentclass[a4paper,10pt]{article}

\usepackage{ctex}

\usepackage[linesnumbered,lined,commentsnumbered,ruled,vlined]{algorithm2e}
\usepackage[top=32mm,right=16mm,bottom=32mm,left=16mm]{geometry}
\usepackage{hyperref}
\hypersetup{
    colorlinks=true,
    linkcolor=blue,
    filecolor=magenta,      
    urlcolor=cyan,
}

\newcommand{\SMR}{SM{\footnotesize A}R{\footnotesize T}}

\title{\SMR}
\author{Sammy}
\date{2018-07-14}

\begin{document}
\maketitle
\tableofcontents

\SMR 由以下3个子算法组成
\begin{itemize}
  \item 客户端操作
  \item 节点的正常运行阶段
  \item 节点发生异常时再同步阶段
\end{itemize}

\section{客户端操作}

\begin{algorithm}
  \DontPrintSemicolon
  \SetKwInOut{In}{input}
  \SetKwInOut{Out}{output}

  \BlankLine
  \(x\leftarrow 0\)\;
  \For{\(i\leftarrow 0\) \KwTo \(8\)} {
    \(x\leftarrow x+i\)\;
  }
  \caption{初始化函数}
\end{algorithm}

\end{document}